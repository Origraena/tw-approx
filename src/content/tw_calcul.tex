% tw_calcul.tex

La preuve de la proposition \ref{propdiestelh} de Diestel a l'avantage d'être
constructive. Ainsi on peut en déduire un algorithme permettant de trouver
une approximation de décomposition arborescente.

\subsection{Algorithme}
\label{algo}
\begin{center}
\begin{algorithm}[H]
\caption{Approximation largeur arborescente}\label{algo_tw}
\algsetup{indent=2em,linenodelimiter= }
\begin{algorithmic}[1]
\REQUIRE{Graphe $G = (V,E)$, $k \in \N$}
\ENSURE{$D$ une décomposition arborescente telle que $width(D) \leq 4k$,\\
	ou $ECHEC$ (et $tw(G) > k$)}
	\STATE $U \leftarrow X : X \subseteq V \wedge |X| = 3k$
	\STATE $D = (T,\{X_t : \forall t \in T\}) \leftarrow (t_0,\{X\})$
	\WHILE{$U \neq V$}
		\STATE Soit $C$ une composante connexe de $G \setminus U$
		\STATE Soit $X \leftarrow \Gamma(C)$ le voisinage de $C$
		\STATE Soit $t \in T : X \subseteq X_t$ le n\oe ud de $T$ contenant $X$
		\IF{$|X| < 3k$}
			\STATE $X' \leftarrow X \cup \{x\} : x \in C \cap \Gamma(X_t)$
			\STATE $U \leftarrow U \cup \{x\}$
		\ELSE
			\IF{$X$ est $(k+1)$-lié}
				\RETURN $ECHEC (tw(G) > k)$
			\ELSE
				\STATE Soient $Y,Z \subseteq V$ les ensembles séparables
				de $X$
				\STATE Soit $S \subseteq C \cup Y \cup Z$ le séparateur de $Y$ et $Z$
				\STATE $X' \leftarrow X \cup S$
				\STATE $U \leftarrow U \cup S$
			\ENDIF
		\ENDIF
		\STATE $D \leftarrow D \cup \{X'\}$
		\STATE $E(T) \leftarrow E(T) \cup (t,t')$
	\ENDWHILE
	\RETURN $D$
\end{algorithmic}
\end{algorithm}
\end{center}

\subsection{Analyse}
\label{analyse}
La décomposition arborescente est produite selon la construction
de la preuve \ref{propdiestelh} (figure \ref{fig_twbuild}).
Si le graphe $G$ admet une largeur arborescente $tw(G) \leq k$,
alors cet algorithme renvoie une décomposition arborescente $D$ de largeur $width(D) \leq 4k$.
C'est à dire que l'algorithme garantit (via la proposition \ref{propdiestel}), que
si la construction de la décomposition échoue, alors la largeur arborescente du graphe est
strictement supérieure à $k$.
De plus, la construction assure que tout au long de l'algorithme, chaque composante de $G \setminus U$
a au plus $3k$ voisins dans $U$, et qu'ils sont tous contenus dans un seul n\oe ud de la décomposition.

En ce qui concerne le temps d'exécution ;
La boucle est itérée au pire $n$ fois (avec $n$ le nombre de sommets de $G$).% et le calcul des
%composantes et de leur voisinage est en $O(n+m) = O(n^2)$.
L'étape de calcul du séparateur est la plus coûteuse, il faut pour 
chaque sous-ensemble $Y,Z \in X : |Y| = |Z| \leq k + 1$ ($3^{|X| = 3k}$) tester
s'ils sont séparables via un algorithme de flot (polynomial en $k$).

Ainsi la complexité en temps de cette approximation est en $O(3^{3k}.n.k)$.

