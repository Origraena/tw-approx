% tw_def.tex

\definition{Décomposition arborescente}
\label{treedec}
	Une décomposition arborescente d'un graphe $G = (V,E)$ est une paire
	$(T,\{X_t : t\in T\}))$
	, avec $T$ un arbre, et $\forall t \in T : V_t \subseteq V)$, respectant
	les trois propriétés suivantes :
	\begin{itemize}
		\item {\em couverture des sommets :}
		$\forall x \in V, \exists t \in T : x \in X_t$            
		\item {\em couverture des arêtes :}
		$\forall (x,y) \in E, \exists t \in T : x,y \in X_t$      
		\item {\em consistance :}
		$\forall x \in V, (\exists t_1,t_2 \in T : x\in X_{t_1} \cap X_{t_2} \rightarrow
	     \forall t\ sur\ le\ chemin\ t_1t_2 \in T, x\in X_t)$
	\end{itemize}

\definition{Largeur d'une décomposition arborescente}
\label{width}
	La largeur d'une décomposition arborescente $(T,\{X_t : t \in T\})$ d'un graphe
	$G = (V,E)$ est définie par : $width(T) = \max\limits_{t \in T}|X_t| - 1$
(voir figure \ref{fig_extreedec}).

% extreedec.tex

\begin{figure}[H]
\begin{center}
\begin{tikzpicture}[]
	\node [vertex] (v0) {};
	\node [vertex] (v1) [below of=v0] {};
	\node [vertex] (v2) [right of=v0] {};
	\node [vertex] (v3) [right of=v1] {};
	\node [vertex] (v4) [right of=v3] {};
	\node [vertex] (v5) [below of=v3] {};
	\node [vertex] (v6) [below of=v4] {};
	\node [vertex] (v7) [below right of=v5] {};
	\node [vertex] (v8) [right of=v4] {};
	\node [vertex] (v9) [above right of=v4] {};
	\node [vertex] (v10) [above left of=v9] {};
	\node [label] (lv0) at (1,0.5) {{\Large \color{black} G}};

	\draw [edge] (v0) -- (v1);
	\draw [edge] (v0) -- (v2);
	\draw [edge] (v1) -- (v3);
	\draw [edge] (v2) -- (v3);
	\draw [edge] (v2) -- (v4);
	\draw [edge] (v2) -- (v9);
	\draw [edge] (v2) -- (v10);
	\draw [edge] (v3) -- (v4);
	\draw [edge] (v3) -- (v5);
	\draw [edge] (v4) -- (v6);
	\draw [edge] (v4) -- (v8);
	\draw [edge] (v4) -- (v9);
	\draw [edge] (v5) -- (v6);
	\draw [edge] (v5) -- (v7);
	\draw [edge] (v6) -- (v7);
	\draw [edge] (v8) -- (v9);
	\draw [edge] (v9) -- (v10);

	\node [treedec] (t0) at (1,-1) {};
	\node [treedec] (t1) at (2.7,-1.3) {};
	\node [treedec] (t2) at (3,-3) {};
	\node [treedec] (t3) at (3.2,-4.5) {};
	\node [treedec] (t4) at (4,-1) {};
	\node [treedec] (t5) at (5.25,-1.5) {};
	\node [treedec] (t6) at (4,0.25) {};
	\node [label] (lt0) at (0.5,-0.5) {{\Large \color{red} T}};

	\draw [treeedge] (t0) -- (t1);
	\draw [treeedge] (t1) -- (t2);
	\draw [treeedge] (t1) -- (t4);
	\draw [treeedge] (t2) -- (t3);
	\draw [treeedge] (t4) -- (t5);
	\draw [treeedge] (t4) -- (t6);

\end{tikzpicture}
\end{center}
\caption{exemple de décomposition arborescente ($width(T) = 3$)}
\label{fig_extreedec}
\end{figure}



\definition{Largeur arborescente}
\label{treewidth}
	La largeur arborescente d'un graphe $G = (V,E)$ est :
	$tw(G) = \min\limits_{\forall T \in \cal{T}_G} width(T)$
	, où $\cal{T}_G$ représente l'ensemble des décompositions
	arborescentes de $G$ .

%%%%\observation{N\oe uds doublons}{
%%%%	Quelle que soit la décomposition $(T,\{X_t : t \in T\})$, si il existe
%%%%	$t_i,t_j \in T : X_{t_i} \subseteq X_{t_j}$ alors la décomposition
%%%%	arborescente $(T' = T \setminus {t_i}, X_t : t \in T'\})$ est valide
%%%%	et a la même largeur que $T$.\\
%%%%	Ainsi, par la suite on ne tiendra compte que de décompositions arborescentes
%%%%	pour lesquelles aucun n\oe ud n'est contenu dans un autre.
%%%%}

\definition{N\oe ud interne}
\label{intnode}
	Soit $(T,\{X_t : t\in T\})$ une décomposition arborescente.
	Tout n\oe ud $t$ est un {\em n\oe ud interne} de $T$ si 
	$\exists t_1 \neq t_2 \in T : (t,t_1),(t,t_2) \in E(T) \wedge X_{t_1},X_{t_2} \subsetneq X_t$.

\definition{Séparation}
\label{defsep}
	Une {\em séparation} d'un graphe $G = (V,E)$ est une paire de deux sous-ensembles de sommets
	$(A,B)$ telle que :
	\begin{itemize}
		\item $A,B \subseteq V$
		\item $A \cup B = V$
		\item il n'existe aucun chemin dans $G \setminus (A\cap B)$ reliant $A \setminus (A \cap B)$
		et $B \setminus (A \cap B)$
	\end{itemize}

\proposition{}
\label{xsep}
	Soient $G = (V,E)$ un graphe et $(T,\{X_t : t\in T\})$ une décomposition arborescente.\\
	Tout n\oe ud interne $X_t$ est un {\em séparateur} de $G$.

\preuve{xsep}
	Soient $G = (V,E)$ un graphe et $(T,\{X_t : t\in T\})$ une décomposition arborescente.
\\
	Soit $t$ un n\oe ud interne de $T$.
\\
	Puisque $T$ est un arbre, $t$ est un séparateur de $T$, on note
	$T_1,\dots,T_\ell$ les composantes connexes de $T\setminus\{t\}$ ;
	% On observe que $\forall i \in [\ell] T_i$ est un sous-arbre de $T$.
	$V_i = \{x \in V : \exists t' \in T_i, x\in X_{t'}\}$ l'ensemble
	des sommets de $G$ présents dans les n\oe uds du sous-arbre $T_i$ ;
	et $G_i = G[V_i\setminus X_t]$
	(voir figure \ref{fig_xsep}).
\\
	Par la couverture des sommets, $\forall x \in V \setminus \{X_t\}$,
	$\exists i \in [\ell] : x \in V(G_i)$.
\\
	{\em Hypothèse :} $\exists i \neq j \in [\ell], \exists x \in G_i, y \in G_j : (x,y) \in E$.
\\
	Par la propriété de couverture des arêtes, il existe $t' \in T$ tel que
	$\{x,y\} \subseteq X_t'$.
\\
	On note $T_x$ (resp $T_y$) le sous-arbre associé à $x$ (resp $y$).
\\
	$t' \in T_x \cap T_y$, donc $T_x \cup T_y$ est connexe.
	De plus, $T_x \cap T_i \neq \emptyset$ et $T_y \cap T_j \neq \emptyset$ puisque
	$x \in V(G_i)$ et $y \in V(G_j)$.
	Or $t$ est un séparateur de $T$ et il existe un chemin de $T_i$ vers $T_j$ 
	passant par $T_x \cap T_y$, donc $t \in T_x \cap T_y$.
\\
	On en conclut que $x,y \in X_t$, ce qui est en contradiction avec l'hypothèse.

	% xseparateur.tex

\begin{figure}[H]
\begin{center}
\begin{tikzpicture}[]
	\node [vertex,fill=blue!50] (v0) {};
	\node [vertex,fill=blue!50] (v1) [below of=v0] {};
	\node [vertex,fill=green!50] (v2) [right of=v0] {};
	\node [vertex,fill=green!50] (v3) [right of=v1] {};
	\node [vertex,fill=green!50] (v4) [right of=v3] {};
	\node [vertex,fill=blue!50] (v5) [below of=v3] {};
	\node [vertex,fill=blue!50] (v6) [below of=v4] {};
	\node [vertex,fill=blue!50] (v7) [below right of=v5] {};
	\node [vertex,fill=blue!50] (v8) [right of=v4] {};
	\node [vertex,fill=blue!50] (v9) [above right of=v4] {};
	\node [vertex,fill=blue!50] (v10) [above left of=v9] {};

	\node [label] (lg1) at (-0.5,-0.5) {{\Large \color{blue} $G_1$}};
	\node [label] (lg2) at (5.5,0) {{\Large \color{blue} $G_2$}};
	\node [label] (lg3) at (2.5,-5) {{\Large \color{blue} $G_3$}};
	\node [label] (lxt) at (1.5,-2.5) {{\Large \color{green!66!black} $X_t$}};

	\draw [giedge] (v0) -- (v1);
	\draw [edge,dotted] (v0) -- (v2);
	\draw [edge,dotted] (v1) -- (v3);
	\draw [edge,dotted,draw=green] (v2) -- (v3);
	\draw [edge,dotted,draw=green] (v2) -- (v4);
	\draw [edge,dotted] (v2) -- (v9);
	\draw [edge,dotted] (v2) -- (v10);
	\draw [edge,dotted,draw=green] (v3) -- (v4);
	\draw [edge,dotted] (v3) -- (v5);
	\draw [edge,dotted] (v4) -- (v6);
	\draw [edge,dotted] (v4) -- (v8);
	\draw [edge,dotted] (v4) -- (v9);
	\draw [giedge] (v5) -- (v6);
	\draw [giedge] (v5) -- (v7);
	\draw [giedge] (v6) -- (v7);
	\draw [giedge] (v8) -- (v9);
	\draw [giedge] (v9) -- (v10);

	\node [treedec] (t0) at (1,-1) {};
	\node [treedec,fill=green!50] (t1) at (2.7,-1.3) {};
	\node [treedec] (t2) at (3,-3) {};
	\node [treedec] (t3) at (3.2,-4.5) {};
	\node [treedec] (t4) at (4,-1) {};
	\node [treedec] (t5) at (5.25,-1.5) {};
	\node [treedec] (t6) at (4,0.25) {};

	\node [label] (lt1) at (0.5,-0.5) {{\Large \color{red} $T_1$}};
	\node [label] (lt2) at (3.5,-0.5) {{\Large \color{red} $T_2$}};
	\node [label] (lt3) at (2.5,-3.5) {{\Large \color{red} $T_3$}};
	\node [label] (ltt) at (2.5,-1) {{\Large \color{green!66!red} $t$}};

	\draw [treeedge,dotted] (t0) -- (t1);
	\draw [treeedge,dotted] (t1) -- (t2);
	\draw [treeedge,dotted] (t1) -- (t4);
	\draw [treeedge] (t2) -- (t3);
	\draw [treeedge] (t4) -- (t5);
	\draw [treeedge] (t4) -- (t6);


\end{tikzpicture}
\end{center}
\caption{illustration preuve \ref{xsep}}
\label{fig_xsep}
\end{figure}



	%	TODO
%\observation{Séparation associée}
%	On remarque que les ensembles séparés sont ceux des sous-arbres de T \ t

\proposition{}
\label{isep}
	Soient $G = (V,E)$ un graphe et $(T,\{X_t : t\in T\})$ une décomposition arborescente.\\
	$\forall (t_1,t_2) \in E(T) : X_{t_1} \cap X_{t_2}$ est un {\em séparateur} de $G$.

\preuve{isep}
	Soient $G = (V,E)$ un graphe et $(T,\{X_t : t\in T\})$ une décomposition arborescente.
	Soient $t_1$ et $t_2$ adjacents dans l'arbre.
	\\
	On construit une décomposition $T'$ de $G$ telle que $V(T') = V(T) \cup \{t'\}$ avec
	$X_{t'} = X_{t_1} \cap X_{t_2}$.
	et $E(T') = E(T) \setminus (t_1,t_2) \cup \{(t_1,t'),(t',t_2)\}$
	(voir figure \ref{fig_isep}).
	\\
	{\em Intuitivement,
	on ajoute un n\oe ud $t'$ dans la décomposition contenant les sommets de $G$
	appartenant à la fois à $t_1$ et à $t_2$,
	puis on coupe l'arête entre $t_1$ et $t_2$ afin d'insérer $t'$ entre ces deux
	n\oe uds dans l'arbre. }
	\\
	La nouvelle décomposition $T'$ respecte
	toujours la couverture des sommets et des arêtes (puisqu'on n'a
	modifié aucun ensemble de sommets $X_t$ correspondant à un n\oe ud de l'arbre).
	Concernant la consistance, pour tout sommet $x \in V$ dont le sous-arbre d'occurences
	dans $T$ passe par l'arête $(t_1,t_2)$, on a $x \in X_{t_1}$ et $x \in X_{t_2}$,
	donc $x \in X_{t_1} \cap X_{t_2}$, et alors dans $T'$, $x \in X_{t'}$.
	\\
	$T'$ est donc une décomposition arborescente valide.
	De plus, il existe $t_1 \neq t_2 \in T : (t',t_1),(t',t_2) \in E(T)$, donc $t'$
	est un n\oe ud interne de l'arbre.
	Enfin d'après la proposition \ref{xsep}, $t'$ étant un sommet interne
	de $T'$, $X_{t'}$ est un séparateur de $G$.

\definition{Séparation correspondante}
	Soient $G = (V,E)$ un graphe, et $(T,\{X_t : t \in T\})$ une décomposition arborescente
	de $G$. Pour chaque arête $(t_i,t_j) \in E(T)$, on note $T_i$ (resp $T_j$) le sous-arbre
	associé à la composante de $T \setminus (t_i,t_j)$ contenant $t_i$ (resp $t_j$).
	Soient les sous-ensembles de sommets $X_i = \bigcup\limits_{t\in T_i}X_t$ et
	$X_j = \bigcup\limits_{t\in T_j}X_t$ associés aux composantes du graphe séparées
	par $X_{t_i} \cap X_{t_j}$.
	
	Toute séparation $(A,B)$ de $G$ avec les ensembles de sommets $A = X_i$ et
	$B = X_j$ est dite {\em séparation correspondante de } $(t_i,t_j)$.

