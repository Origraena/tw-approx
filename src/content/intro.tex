% intro.tex

%Bien qu'il ait été démontré que décider si 

Les décompositions arborescentes sont très utiles, permettant
de modifier la structure d'un graphe pour permettre de résoudre
des problèmes difficiles dans le cas général, en ramenant la topologie
du graphe à un {\em arbre}.

De plus, de nombreux problèmes appartenant à la classe NP-difficile
peuvent atteindre des complexités polynomiales pour une valeur de
largeur arborescente fixée, par exemple {\em Ensemble-Indépendant},
et {\em Cycle Hamiltonien}.

Cependant, il a été démontré par Arnborg, Corneil et Proskurowski que le
problème de décider si la largeur arborescente d'un graphe est inférieure
ou égale à une valeur donnée est NP-difficile \cite{arnborg}.
Bodlaender quant à lui, prouva l'existence d'un 
algorithme de complexité $O(2^{k^3}.n)$ permettant de répondre
à cette question \cite{bodlaender}.

Puisque la largeur arborescente est avant tout utilisée comme
paramètre d'autres problèmes, Diestel \cite{diestel} proposa une approximation
avec garantie permettant de construire une décomposition de largeur
bornée.

Tout d'abord la largeur arborescente est définie formellement, et quelques propriétés
sont établies. S'en suit deux propositions fortes permettant de construire une
décomposition arborescente $4$-approchée en temps polynomial lorsque paramétrée par
la valeur de la solution.
Enfin un algorithme basé sur cette construction est donné, et sa complexité analysée.


